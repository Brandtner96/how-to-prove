%%%%%%%%%%%%%%%%%%%%%%%%%%%%%%%%%%%%%%%%%
% Minimalists Report template
% Author: Sibi <sibi@psibi.in>
%%%%%%%%%%%%%%%%%%%%%%%%%%%%%%%%%%%%%%%%%
\documentclass{article}
\usepackage{graphicx}
\usepackage{verbatim}
\usepackage{amsmath}
\usepackage{amsfonts}
\usepackage{amssymb}
\begin{document}
\title{Chapter 2 (Section 2.3)}
\author{Sibi}
\date{\today}
\maketitle
\newpage

\section{Problem 1}

\begin{center}
(1.1) $F \subseteq \wp(A)$
\end{center}  
\begin{align*}
\forall x (x \in F \implies x \in \wp(A)) \\
\forall x (x \in F \implies x \subseteq A) \\
\forall x (x \in F \implies \forall y (y \in x \implies y \in A)) 
\end{align*}

\begin{center}
(1.2) $A \subseteq \{2n + 1 | n \in \mathbb{N}\}$ \\
\end{center}
\begin{align*}
\forall x (x \in A \implies x \in \{2n + 1 | n \in \mathbb{N}\}) \\
\forall x (x \in A \implies \exists n \in \mathbb{N} (x=2n+1))
\end{align*}

\begin{center}
(1.3) $\{n^2 + n + 1 | n \in \mathbb{N} \} \subseteq \{2n + 1 | n \in \mathbb{N} \}$ \\
\end{center}
\begin{align*}
\forall x (x \in \left\{n^2 + n + 1 \mid n \in \mathbb{N} \right\}) \implies (x \in \left\{2n + 1 \mid n \in \mathbb{N} \right\}) \\
\forall x (\exists n \in \mathbb{N} (x = n^2 + n + 1)) \implies (\exists n \in \mathbb{N} (x=2n+1)) \\
\end{align*}

\begin{center}
(1.4) $\wp(\cup_{i \in I} A_i) \nsubseteq \cup_{i \in I} \wp(A_i)$
\end{center}
\begin{align*}
\exists x(x \in \wp(\cup_{i \in I} A_i) \land x \notin \cup_{i \in I} \wp(A_i)) \\
\exists x(x \subseteq \cup_{i \in I} A_i \land x \notin \cup_{i \in I} \wp(A_i)) \\
\exists x( \forall y (y \in x \implies y \in \cup_{i \in I} A_i) \land
x \notin \cup_{i \in I} \wp(A_i)) \\
\exists x( \forall y (y \in x \implies \exists i \in I(y \in A_i))
\land x \notin \cup_{i \in I} \wp(A_i)) \\
\exists x( \forall y (y \in x \implies \exists i \in I(y \in A_i))
\land \neg (x \in \cup_{i \in I} \wp(A_i))) \\
\exists x( \forall y (y \in x \implies \exists i \in I(y \in A_i))
\land \neg (\exists i \in I(x \in \wp(A_i)))) \\
\exists x( \forall y (y \in x \implies \exists i \in I(y \in A_i))
\land \neg (\exists i \in I(x \subseteq A_i))) \\
\exists x( \forall y (y \in x \implies \exists i \in I(y \in A_i))
\land \neg (\exists i \in I(\forall z (z \in x \implies z \in A_i)))) \\
\exists x( \forall y (y \in x \implies \exists i \in I(y \in A_i))
\land (\forall i \in I(\exists z \neg(z \in x \implies z \in A_i)))) \\
\exists x( \forall y (y \in x \implies \exists i \in I(y \in A_i))
\land (\forall i \in I(\exists z (z \in x \land \neg(z \in A_i))))) \\
\end{align*}

\section{Problem 2}
\begin{center}
(2.1) $x \in \cup F \setminus \cup G $
\end{center}
\begin{align*}
x \in \cup F \land x \notin \cup G \\
\exists A \in F(x \in A) \land \exists A \in G(x \notin A)
\end{align*}

\begin{center}
(2.2) $\left\{ x \in B \mid x \notin C\right\} \in \wp(A)$ 
\end{center}
\begin{align*}
\left\{ x \in B \mid x \notin C\right\} \subseteq A  \\
\forall y(y \in \left\{ x \in B \mid x \notin C\right\} \implies y \in A) \\
\forall y(y \in  B \land y \notin C \implies y \in A) \\
\end{align*}

\begin{center}
(2.3) $ x \in \cap_{i \in I}(A_i \cup B_i)$ 
\end{center}
\begin{align*}
\forall i \in I(x \in (A_i \cup B_i)) \\
\forall i \in I(x \in A_i \lor x \in B_i)
\end{align*}

\begin{center}
(2.4) $x \in (\cap_{i \in I}A_i) \cup (\cap_{i \in I}B_i)$ 
\end{center}
\begin{align*}
\forall i \in I(x \in A_i)) \lor (\forall i \in I(x \in B_i)
\end{align*}

\section{Problem 3}
\begin{align*}
  \left\{\left\{\varnothing\right\}, \varnothing \right\}
\end{align*}

\section{Problem 4}
\begin{align*}
  F = \left\{ \left\{red, green, blue \right\}, \left\{orange, red,
  green\right\}, \left\{purple, red, green, blue \right\} \right\} \\
  \cup F = \left\{ red, green, blue, orange, purple\right\} \\
  \cap F = \left\{ red, green\right\} \\
\end{align*}

\section{Problem 5}
\begin{align*}
  F = \left\{ \left\{3,7,12 \right\}, \left\{5,7,16
  \right\}, \left\{5,12,23 \right\} \right\} \\
  \cup F = \left\{ 3,7,12,5,16,23 \right\} \\
  \cap F = \varnothing \\
\end{align*}

\section{Problem 6}
\begin{center}
  $I = \left\{2,3,4,5\right\}$ \\
  $A_i = \left\{i, i+1, i-1, 2*i\right\}$ \\
\end{center}
\begin{align*}
  A_2 = \left\{2,3,1,4\right\} \\
  A_3 = \left\{3,4,2,6\right\} \\
  A_4 = \left\{4,5,3,8\right\} \\
  A_5 = \left\{5,6,4,10\right\} \\
  \cap_{i \in I}A_i = \left\{4 \right\} \\
  \cup_{i \in I}A_i = \left\{1,2,3,4,5,6,8,10 \right\} \\
\end{align*}

\section{Problem 7}
Too lazy to figure out when they live.
Or if you want a more reasonable answer: Left as an exercise to the reader :P
\section{Problem 8}
\begin{center}
  $I = \left\{2,3\right\}$ \\
  $A_i = \left\{ i,2i \right\}$ \\
  $B_i = \left\{ i, i + 1 \right\}$ \\
\end{center}
\begin{align*}
  A_2 = \left\{ 2, 4 \right\} \\
  A_3 = \left\{ 3, 6 \right\} \\
  B_2 = \left\{ 2, 3 \right\} \\
  B_3 = \left\{ 3, 4 \right\} \\ \\
  \cap_{i \in I}(A_i \cup B_i) = \left\{3,4 \right\} \\
  (\cap_{i \in I}A_i) \cup (\cap_{i \in I}B_i) = \left\{ 3\right\} \\ \\
  No they are not equivalent. 2(c) and 2(d) are not equivalent to each other.
\end{align*}

\section{Problem 9}
\begin{align*}
  Let X = \cup_{i \in I} (A_i \cap B_i) \\
  Let Y = (\cup_{i \in I}A_i) \cap (\cup_{i \in  I} B_i) \\ \\
  x \in X is equivalent to x \in \cup_{i \in I} (A_i \cap B_i) \\
  \exists i \in I(x \in (A_i \cap B_i))) \\
  \exists i \in I(x \in A_i \land x \in B_i) \\ \\
  x \in Y is equivalent to x \in (\cup_{i \in I}A_i) \cap (\cup_{i \in  I} B_i) \\
  (x \in \cup_{i \in I}A_i) \land (x \in \cup_{i \in  I} B_i) \\
  \exists i \in I(x \in A_i) \land \exists i \in I (x \in B_i) \\ \\
\end{align*}
Now clearly they are not equivalent to each other. A trivial example of them would be:
$Let I = \left\{1,2\right\}$ \\
$A_i = \left\{i\right\}$ \\
$B_i = \left\{2i\right\}$ \\

\section{Problem 10}
\begin{align*}
  x \in \wp(A \cap B) \\
  x \subseteq (A \cap B) \\
  \forall y (y \in x \implies y \in (A \cap B)) \\
  \forall y (y \in x \implies y \in A \land y \in B) \\
  \forall y (y \notin x \lor (y \in A \land y \in B)) \\
  \forall y ((y \notin x \lor y \in A) \land (y \notin x \lor y \in B)) \\
  \forall y(y \in x \implies y \in A) \land \forall y(y \in x \implies y \in B) \\ \\
  x \in \wp(A) \cap \wp(B) \\
  x \in \wp(A) \land x \in \wp(B) \\
  x \subseteq A \land x \subseteq B \\
  \forall y(y \in x \implies y \in A) \land \forall y(y \in x \implies y \in B) \\
\end{align*}
%% \left\{ \right\} 
\section{Problem 11}
\begin{align*}
 A = \left\{1 \right\} \\
 B = \left\{2 \right\} \\
 \wp(A \cup B) = \left\{\left\{1,2 \right\},\left\{1 \right\}, \left\{2 \right\}, \varnothing  \right\} \\
 \wp(A) \cup \wp(B) = \left\{ \left\{1 \right\},\left\{2 \right\}, \varnothing  \right\} 
\end{align*}

\section{Problem 12}
Problem (a)
\begin{align*}
  \cup_{i \in I}(A_i \cup B_i) = (\cup_{i \in I}A_i) \cup (\cup_{i \in  I}B_i) \\\\
  x \in \cup_{i \in I}(A_i \cup B_i) \\
  \exists i \in I(x \in A_i \cup B_i) \\
  \exists i \in I(x \in A_i \lor x \in B_i)\\\\
  x \in (\cup_{i \in I}A_i) \cup (\cup_{i \in I}B_i) \\
  x \in (\cup_{i \in I}A_i) \lor x \in (\cup_{i \in I}B_i) \\
  \exists i \in I(x \in A_i) \lor \exists i \in I(x \in B_i) \\
  \exists i \in I(x \in A_i \lor x \in B_i) \\
\end{align*}
Problem (b)
\begin{align*}
  (\cap F) \cap (\cap G) = \cap(F \cup G)\\
  x \in (\cap F) \cap (\cap G) \\
  x \in (\cap F) \land x \in (\cap G) \\
  \forall A \in F(x \in A) \land \forall A \in G(x \in A)\\\\
  x \in \cap(F \cup G) \\
  \forall A \in (F \cup G)(x \in A)\\
  \forall A(A \in (F \cup G) \implies x \in A) \\
  \forall A(A \in F \lor A \in G \implies x \in A)\\
  \forall A((A \notin F \land A \notin G) \lor x \in A)\\
  \forall A((A \notin F \lor x \in A) \land (A \notin G \lor x \in A))\\
  \forall A(A \notin F \lor x \in A) \land \forall A(A \notin G \lor x \in A)\\
  \forall A(A \in F \implies x \in A) \land \forall A(A \in G \implies x \in A)\\
  \forall A \in F(x \in A) \land \forall A \in G(x \in A)\\
\end{align*}
Problem (c)
\begin{align*}
  \cap_{i \in I}(A_i \setminus B_i) = (\cap_{i \in I}A_i) \setminus (\cup_{i \in I}B_i)\\
  x \in \cap_{i \in I}(A_i \setminus B_i) \\
  \forall i \in I(x \in (A_i \setminus B_i)) \\
  \forall i \in I(x \in A_i \land x \notin B_i)\\
  \forall i \in I(x \in A_i) \land \forall i \in I(x \notin B_i)\\\\
  x \in (\cap_{i \in I}A_i) \setminus (\cup_{i \in I}B_i)\\
  x \in (\cap_{i \in I}A_i) \land x \notin (\cup_{i \in I}B_i) \\
  \forall i \in I(x \in A_i) \land \neg (x \in \cup_{i \in I}B_i) \\
  \forall i \in I(x \in A_i) \land \neg (\exists i \in I(x \in B_i)) \\
  \forall i \in I(x \in A_i) \land \forall i \in I (x \notin B_i)
\end{align*}

\section{Problem 13}
$$ I = \{1,2\} $$
$$ J = \{3,4\} $$
$$ A_{i,j} = \{i, j, i+j\} $$
Problem (a)
\begin{align}
  B_j = U_{i \in I}A_{i,j} = A_{1,j} \cup A_{2,j} \\
  B_3 = A_{1,3} \cup A_{2,3} \\
  = \{ 1,3,4 \} \cup \{ 2,3,5 \} \\
  = \{1,2,3,4,5\} \\ \\
  B_4 = A_{1,4} \cup A_{2,4} \\
  = \{1,4,5\} \cup \{2,4,6\} \\
  = \{1,2,4,5,6\} \\
\end{align}
Problem (b)
\begin{align*}
  \cap_{j \in J}B_j \\
  B_3 \cap B_4 \\
  \{1,2,4,5\} \\
\end{align*}
Problem (c)
\begin{align*}
  \cup_{i \in I}(\cap_{j \in J}A_{i,j}) \\
  \cup_{i \in I}(A_{i,3} \cap A_{i,4}) \\
  (A_{1,3} \cap A_{1,4}) \cup (A_{2,3} \cap A_{2,4}) \\
  (\{1,3,4\} \cap \{1,4,5\}) \cup (\{2,3,5\} \cap \{2,4,6\}) \\
  \{1,4\} \cup \{2\} \\
  \{1,2,4\} \\
\end{align*}
No, they are not equivalent. \\
Problem (d)
\begin{align*}
  x \in \cap_{j \in J}(\cup_{i \in I}A_{i,j}) \\
  \forall j \in J(x \in \cup_{i \in I}A_{i,j})\\
  \forall j \in J(\exists i \in I(x \in A_{i,j}))\\ \\
  x \in \cup_{i \in I}(\cap_{j \in J}A_{i,j}) \\
  \exists i \in I(x \in (\cap_{j \in J}A_{i,j})) \\
  \exists i \in I(\forall j \in J(x \in A_{i,j}))\\
\end{align*}
  No, they are not equivalent.
\section{Problem 14}
Problem (a)
\begin{align*}
  x \in \cup F \\
  \exists A \in F(x \in A)  \\
  \exists A (A \in F \land x \in A) \\
\end{align*}
Now if $F$ is $\emptyset$ then the statement is obviously false. There
is no $A$ that will satisfy that logical form. Hence $\cup \emptyset =
\emptyset$ \\

Problem (b)
\begin{align*}
  x \in \cap F \\
  \forall A \in F(x \in A) \\
  \forall x (A \in F \implies x \in A) \\
\end{align*}
Now if $F$ is $\emptyset$, then the antecedent is false. But the
logical form is vacously true. So every element in $U$ will satisfy
that. Therefore, $\cap \emptyset = U$
\end{document}

